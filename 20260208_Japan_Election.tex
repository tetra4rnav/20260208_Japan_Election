% Options for packages loaded elsewhere
\PassOptionsToPackage{unicode}{hyperref}
\PassOptionsToPackage{hyphens}{url}
\documentclass[
]{article}
\usepackage{xcolor}
\usepackage[margin=1in]{geometry}
\usepackage{amsmath,amssymb}
\setcounter{secnumdepth}{-\maxdimen} % remove section numbering
\usepackage{iftex}
\ifPDFTeX
  \usepackage[T1]{fontenc}
  \usepackage[utf8]{inputenc}
  \usepackage{textcomp} % provide euro and other symbols
\else % if luatex or xetex
  \usepackage{unicode-math} % this also loads fontspec
  \defaultfontfeatures{Scale=MatchLowercase}
  \defaultfontfeatures[\rmfamily]{Ligatures=TeX,Scale=1}
\fi
\usepackage{lmodern}
\ifPDFTeX\else
  % xetex/luatex font selection
\fi
% Use upquote if available, for straight quotes in verbatim environments
\IfFileExists{upquote.sty}{\usepackage{upquote}}{}
\IfFileExists{microtype.sty}{% use microtype if available
  \usepackage[]{microtype}
  \UseMicrotypeSet[protrusion]{basicmath} % disable protrusion for tt fonts
}{}
\makeatletter
\@ifundefined{KOMAClassName}{% if non-KOMA class
  \IfFileExists{parskip.sty}{%
    \usepackage{parskip}
  }{% else
    \setlength{\parindent}{0pt}
    \setlength{\parskip}{6pt plus 2pt minus 1pt}}
}{% if KOMA class
  \KOMAoptions{parskip=half}}
\makeatother
\usepackage{color}
\usepackage{fancyvrb}
\newcommand{\VerbBar}{|}
\newcommand{\VERB}{\Verb[commandchars=\\\{\}]}
\DefineVerbatimEnvironment{Highlighting}{Verbatim}{commandchars=\\\{\}}
% Add ',fontsize=\small' for more characters per line
\usepackage{framed}
\definecolor{shadecolor}{RGB}{248,248,248}
\newenvironment{Shaded}{\begin{snugshade}}{\end{snugshade}}
\newcommand{\AlertTok}[1]{\textcolor[rgb]{0.94,0.16,0.16}{#1}}
\newcommand{\AnnotationTok}[1]{\textcolor[rgb]{0.56,0.35,0.01}{\textbf{\textit{#1}}}}
\newcommand{\AttributeTok}[1]{\textcolor[rgb]{0.13,0.29,0.53}{#1}}
\newcommand{\BaseNTok}[1]{\textcolor[rgb]{0.00,0.00,0.81}{#1}}
\newcommand{\BuiltInTok}[1]{#1}
\newcommand{\CharTok}[1]{\textcolor[rgb]{0.31,0.60,0.02}{#1}}
\newcommand{\CommentTok}[1]{\textcolor[rgb]{0.56,0.35,0.01}{\textit{#1}}}
\newcommand{\CommentVarTok}[1]{\textcolor[rgb]{0.56,0.35,0.01}{\textbf{\textit{#1}}}}
\newcommand{\ConstantTok}[1]{\textcolor[rgb]{0.56,0.35,0.01}{#1}}
\newcommand{\ControlFlowTok}[1]{\textcolor[rgb]{0.13,0.29,0.53}{\textbf{#1}}}
\newcommand{\DataTypeTok}[1]{\textcolor[rgb]{0.13,0.29,0.53}{#1}}
\newcommand{\DecValTok}[1]{\textcolor[rgb]{0.00,0.00,0.81}{#1}}
\newcommand{\DocumentationTok}[1]{\textcolor[rgb]{0.56,0.35,0.01}{\textbf{\textit{#1}}}}
\newcommand{\ErrorTok}[1]{\textcolor[rgb]{0.64,0.00,0.00}{\textbf{#1}}}
\newcommand{\ExtensionTok}[1]{#1}
\newcommand{\FloatTok}[1]{\textcolor[rgb]{0.00,0.00,0.81}{#1}}
\newcommand{\FunctionTok}[1]{\textcolor[rgb]{0.13,0.29,0.53}{\textbf{#1}}}
\newcommand{\ImportTok}[1]{#1}
\newcommand{\InformationTok}[1]{\textcolor[rgb]{0.56,0.35,0.01}{\textbf{\textit{#1}}}}
\newcommand{\KeywordTok}[1]{\textcolor[rgb]{0.13,0.29,0.53}{\textbf{#1}}}
\newcommand{\NormalTok}[1]{#1}
\newcommand{\OperatorTok}[1]{\textcolor[rgb]{0.81,0.36,0.00}{\textbf{#1}}}
\newcommand{\OtherTok}[1]{\textcolor[rgb]{0.56,0.35,0.01}{#1}}
\newcommand{\PreprocessorTok}[1]{\textcolor[rgb]{0.56,0.35,0.01}{\textit{#1}}}
\newcommand{\RegionMarkerTok}[1]{#1}
\newcommand{\SpecialCharTok}[1]{\textcolor[rgb]{0.81,0.36,0.00}{\textbf{#1}}}
\newcommand{\SpecialStringTok}[1]{\textcolor[rgb]{0.31,0.60,0.02}{#1}}
\newcommand{\StringTok}[1]{\textcolor[rgb]{0.31,0.60,0.02}{#1}}
\newcommand{\VariableTok}[1]{\textcolor[rgb]{0.00,0.00,0.00}{#1}}
\newcommand{\VerbatimStringTok}[1]{\textcolor[rgb]{0.31,0.60,0.02}{#1}}
\newcommand{\WarningTok}[1]{\textcolor[rgb]{0.56,0.35,0.01}{\textbf{\textit{#1}}}}
\usepackage{graphicx}
\makeatletter
\newsavebox\pandoc@box
\newcommand*\pandocbounded[1]{% scales image to fit in text height/width
  \sbox\pandoc@box{#1}%
  \Gscale@div\@tempa{\textheight}{\dimexpr\ht\pandoc@box+\dp\pandoc@box\relax}%
  \Gscale@div\@tempb{\linewidth}{\wd\pandoc@box}%
  \ifdim\@tempb\p@<\@tempa\p@\let\@tempa\@tempb\fi% select the smaller of both
  \ifdim\@tempa\p@<\p@\scalebox{\@tempa}{\usebox\pandoc@box}%
  \else\usebox{\pandoc@box}%
  \fi%
}
% Set default figure placement to htbp
\def\fps@figure{htbp}
\makeatother
\setlength{\emergencystretch}{3em} % prevent overfull lines
\providecommand{\tightlist}{%
  \setlength{\itemsep}{0pt}\setlength{\parskip}{0pt}}
\usepackage{luatexja}
\usepackage{luatexja-fontspec}
\setmainfont{Times New Roman}
\setmainjfont{Yu Mincho}
\usepackage{fancyhdr}
\pagestyle{fancy}
\fancyhf{}
\fancyhead[R]{Matt Sakamaki}
\renewcommand{\headrulewidth}{0.4pt}
\AtBeginDocument{\thispagestyle{fancy}}
\usepackage{bookmark}
\IfFileExists{xurl.sty}{\usepackage{xurl}}{} % add URL line breaks if available
\urlstyle{same}
\hypersetup{
  pdftitle={2026年 衆議院議員選挙結果の分析},
  pdfauthor={Matt Sakamaki - msakama2@jh.edu},
  hidelinks,
  pdfcreator={LaTeX via pandoc}}

\title{2026年 衆議院議員選挙結果の分析}
\author{Matt Sakamaki -
\href{mailto:msakama2@jh.edu}{\nolinkurl{msakama2@jh.edu}}}
\date{}

\begin{document}
\maketitle

\begin{Shaded}
\begin{Highlighting}[]
\FunctionTok{library}\NormalTok{(showtext)}
\end{Highlighting}
\end{Shaded}

\begin{verbatim}
## Loading required package: sysfonts
\end{verbatim}

\begin{verbatim}
## Loading required package: showtextdb
\end{verbatim}

\begin{Shaded}
\begin{Highlighting}[]
\FunctionTok{library}\NormalTok{(sysfonts)}

\FunctionTok{library}\NormalTok{(tidyverse)}
\end{Highlighting}
\end{Shaded}

\begin{verbatim}
## -- Attaching core tidyverse packages ------------------------ tidyverse 2.0.0 --
## v dplyr     1.1.4     v readr     2.1.5
## v forcats   1.0.0     v stringr   1.5.2
## v ggplot2   4.0.0     v tibble    3.3.0
## v lubridate 1.9.4     v tidyr     1.3.1
## v purrr     1.1.0
\end{verbatim}

\begin{verbatim}
## -- Conflicts ------------------------------------------ tidyverse_conflicts() --
## x dplyr::filter() masks stats::filter()
## x dplyr::lag()    masks stats::lag()
## i Use the conflicted package (<http://conflicted.r-lib.org/>) to force all conflicts to become errors
\end{verbatim}

\begin{Shaded}
\begin{Highlighting}[]
\FunctionTok{library}\NormalTok{(dplyr)}

\FunctionTok{showtext\_auto}\NormalTok{()}

\NormalTok{df}\OtherTok{\textless{}{-}}\FunctionTok{read.csv}\NormalTok{(}\StringTok{"partybalot.csv"}\NormalTok{)}
\NormalTok{df}
\end{Highlighting}
\end{Shaded}

\begin{verbatim}
##   year balot_jimin balot_komei batot_ishin balot_yotousum seat_jimin seat_komei
## 1 2014    25461449      765390           0       26226839        222          9
## 2 2017    26500777      832453           0       27333230        215          8
## 3 2021    27626236      872931           0       28499167        189          9
## 4 2024    20867762      730401           0       21598163        132          4
## 5 2026    27789183           0     4943331       32732514        249          0
##   seat_ishin seat_yotousum seat_total  n_balot   n_voter p_vote n_osbalot
## 1          0           231        295 54743087 103962784   0.53     19256
## 2          0           223        289 56952674 106091229   0.54     21462
## 3          0           198        289 58901616 105320523   0.56     19531
## 4          0           136        289 55935743 103880749   0.54     17288
## 5         20           269        289 58062807 103211224   0.56     28966
##   n_osvoters p_osvote
## 1     104677     0.18
## 2     100405     0.21
## 3      96664     0.20
## 4      95472     0.18
## 5     103380     0.28
\end{verbatim}

\begin{Shaded}
\begin{Highlighting}[]
\FunctionTok{str}\NormalTok{(df)}
\end{Highlighting}
\end{Shaded}

\begin{verbatim}
## 'data.frame':    5 obs. of  16 variables:
##  $ year          : num  2014 2017 2021 2024 2026
##  $ balot_jimin   : num  25461449 26500777 27626236 20867762 27789183
##  $ balot_komei   : num  765390 832453 872931 730401 0
##  $ batot_ishin   : num  0 0 0 0 4943331
##  $ balot_yotousum: num  26226839 27333230 28499167 21598163 32732514
##  $ seat_jimin    : num  222 215 189 132 249
##  $ seat_komei    : num  9 8 9 4 0
##  $ seat_ishin    : num  0 0 0 0 20
##  $ seat_yotousum : num  231 223 198 136 269
##  $ seat_total    : num  295 289 289 289 289
##  $ n_balot       : num  54743087 56952674 58901616 55935743 58062807
##  $ n_voter       : num  1.04e+08 1.06e+08 1.05e+08 1.04e+08 1.03e+08
##  $ p_vote        : num  0.53 0.54 0.56 0.54 0.56
##  $ n_osbalot     : num  19256 21462 19531 17288 28966
##  $ n_osvoters    : num  104677 100405 96664 95472 103380
##  $ p_osvote      : num  0.18 0.21 0.2 0.18 0.28
\end{verbatim}

\section{1}\label{section}

\begin{quote}
自民党得票率(小選挙区)と獲得議席数の関係
\end{quote}

\begin{Shaded}
\begin{Highlighting}[]
\NormalTok{df }\OtherTok{\textless{}{-}}\NormalTok{ df }\SpecialCharTok{\%\textgreater{}\%}
  \FunctionTok{mutate}\NormalTok{(}
    \AttributeTok{balot\_jimin =} \FunctionTok{as.numeric}\NormalTok{(}\FunctionTok{as.character}\NormalTok{(balot\_jimin)),}
    \AttributeTok{n\_voter     =} \FunctionTok{as.numeric}\NormalTok{(}\FunctionTok{as.character}\NormalTok{(n\_voter))}
\NormalTok{  ) }\SpecialCharTok{\%\textgreater{}\%}
  \FunctionTok{mutate}\NormalTok{(}
    \AttributeTok{p\_jimin =}\NormalTok{ balot\_jimin }\SpecialCharTok{/}\NormalTok{ n\_balot,}
    \AttributeTok{p\_seat\_jimin =}\NormalTok{ seat\_jimin }\SpecialCharTok{/}\NormalTok{ seat\_total}
\NormalTok{  )}

\NormalTok{plot1 }\OtherTok{\textless{}{-}} \FunctionTok{ggplot}\NormalTok{(}
\NormalTok{  df,}
  \FunctionTok{aes}\NormalTok{(}
    \AttributeTok{x=}\NormalTok{p\_seat\_jimin,}
    \AttributeTok{y=}\NormalTok{p\_jimin,}
    \AttributeTok{label=}\NormalTok{year}
\NormalTok{  )}
\NormalTok{)}\SpecialCharTok{+}
\FunctionTok{geom\_point}\NormalTok{(}\AttributeTok{size =} \DecValTok{3}\NormalTok{) }\SpecialCharTok{+}
\FunctionTok{geom\_smooth}\NormalTok{(}
  \AttributeTok{method =} \StringTok{"lm"}\NormalTok{,}
  \AttributeTok{se =} \ConstantTok{FALSE}\NormalTok{,}
  \AttributeTok{linewidth =} \FloatTok{0.7}\NormalTok{,}
  \AttributeTok{fullrange =} \ConstantTok{TRUE}
\NormalTok{) }\SpecialCharTok{+}
\FunctionTok{geom\_text}\NormalTok{(}
  \AttributeTok{vjust =} \SpecialCharTok{{-}}\DecValTok{1}\NormalTok{,}
  \AttributeTok{size =} \DecValTok{3}
\NormalTok{) }\SpecialCharTok{+}
  
\FunctionTok{scale\_x\_continuous}\NormalTok{(}
    \AttributeTok{limits =} \FunctionTok{c}\NormalTok{(}\DecValTok{0}\NormalTok{, }\DecValTok{1}\NormalTok{),}
    \AttributeTok{labels =}\NormalTok{ scales}\SpecialCharTok{::}\NormalTok{percent}
\NormalTok{) }\SpecialCharTok{+}
\FunctionTok{scale\_y\_continuous}\NormalTok{(}
    \AttributeTok{limits =} \FunctionTok{c}\NormalTok{(}\DecValTok{0}\NormalTok{, }\DecValTok{1}\NormalTok{),}
    \AttributeTok{labels =}\NormalTok{ scales}\SpecialCharTok{::}\NormalTok{percent}
\NormalTok{) }\SpecialCharTok{+}
  
\FunctionTok{labs}\NormalTok{(}
  \AttributeTok{title =} \StringTok{"衆院選 自民党得票率(小選挙区)と議席獲得率の関係"}\NormalTok{,}
  \AttributeTok{x =} \StringTok{"自民党 議席獲得率(小選挙区)"}\NormalTok{,}
  \AttributeTok{y =} \StringTok{"自民党 得票率 (得票数/投票数)"}\NormalTok{,}
  \AttributeTok{caption =} \StringTok{"※ データ点が少ないため回帰は参考程度(外挿)"}
\NormalTok{) }\SpecialCharTok{+}
\FunctionTok{theme\_minimal}\NormalTok{(}\AttributeTok{base\_size =} \DecValTok{12}\NormalTok{)}

\NormalTok{plot1}
\end{Highlighting}
\end{Shaded}

\begin{verbatim}
## `geom_smooth()` using formula = 'y ~ x'
\end{verbatim}

\begin{verbatim}
## Warning: The following aesthetics were dropped during statistical transformation: label.
## i This can happen when ggplot fails to infer the correct grouping structure in
##   the data.
## i Did you forget to specify a `group` aesthetic or to convert a numerical
##   variable into a factor?
\end{verbatim}

\pandocbounded{\includegraphics[keepaspectratio]{20260208_Japan_Election_files/figure-latex/unnamed-chunk-2-1.pdf}}

\section{2.}\label{section-1}

\begin{quote}
投票率(全体)と在外投票率との関係
\end{quote}

\begin{Shaded}
\begin{Highlighting}[]
\NormalTok{df }\OtherTok{\textless{}{-}}\NormalTok{ df }\SpecialCharTok{\%\textgreater{}\%}
  \FunctionTok{mutate}\NormalTok{(}
    \AttributeTok{p\_vote\_precise =}\NormalTok{ n\_balot }\SpecialCharTok{/}\NormalTok{ n\_voter,}
    \AttributeTok{p\_osvote\_precise =}\NormalTok{ n\_osbalot }\SpecialCharTok{/}\NormalTok{ n\_osvoters}
\NormalTok{  )}

\NormalTok{plot2 }\OtherTok{\textless{}{-}} \FunctionTok{ggplot}\NormalTok{(}
\NormalTok{  df,}
  \FunctionTok{aes}\NormalTok{(}
    \AttributeTok{x=}\NormalTok{p\_vote\_precise,}
    \AttributeTok{y=}\NormalTok{p\_osvote\_precise,}
    \AttributeTok{label=}\NormalTok{year}
\NormalTok{  )}
\NormalTok{)}\SpecialCharTok{+}

\FunctionTok{geom\_point}\NormalTok{(}\AttributeTok{size =} \DecValTok{3}\NormalTok{) }\SpecialCharTok{+}
\FunctionTok{geom\_smooth}\NormalTok{(}
  \AttributeTok{method =} \StringTok{"lm"}\NormalTok{,}
  \AttributeTok{se =} \ConstantTok{FALSE}\NormalTok{,}
  \AttributeTok{linewidth =} \FloatTok{0.7}\NormalTok{,}
  \AttributeTok{fullrange =} \ConstantTok{TRUE}
\NormalTok{) }\SpecialCharTok{+}
\FunctionTok{geom\_text}\NormalTok{(}
  \AttributeTok{vjust =} \SpecialCharTok{{-}}\DecValTok{1}\NormalTok{,}
  \AttributeTok{size =} \DecValTok{3}
\NormalTok{) }\SpecialCharTok{+}

\FunctionTok{labs}\NormalTok{(}
  \AttributeTok{title =} \StringTok{"衆院選 投票率(小選挙区・全体)と在外投票率との関係"}\NormalTok{,}
  \AttributeTok{x =} \StringTok{"投票率"}\NormalTok{,}
  \AttributeTok{y =} \StringTok{"在外投票率"}\NormalTok{,}
  \AttributeTok{caption =} \StringTok{"※ データ点が少ないため回帰は参考程度(外挿)"}
\NormalTok{) }\SpecialCharTok{+}
\FunctionTok{theme\_minimal}\NormalTok{(}\AttributeTok{base\_size =} \DecValTok{12}\NormalTok{)}

\NormalTok{plot2}
\end{Highlighting}
\end{Shaded}

\begin{verbatim}
## `geom_smooth()` using formula = 'y ~ x'
\end{verbatim}

\begin{verbatim}
## Warning: The following aesthetics were dropped during statistical transformation: label.
## i This can happen when ggplot fails to infer the correct grouping structure in
##   the data.
## i Did you forget to specify a `group` aesthetic or to convert a numerical
##   variable into a factor?
\end{verbatim}

\pandocbounded{\includegraphics[keepaspectratio]{20260208_Japan_Election_files/figure-latex/unnamed-chunk-3-1.pdf}}
\#\# 検定

\begin{Shaded}
\begin{Highlighting}[]
\CommentTok{\#pearson}
\FunctionTok{cor.test}\NormalTok{(df}\SpecialCharTok{$}\NormalTok{p\_vote\_precise, df}\SpecialCharTok{$}\NormalTok{p\_osvote\_precise, }\AttributeTok{method =} \StringTok{"pearson"}\NormalTok{)}
\end{Highlighting}
\end{Shaded}

\begin{verbatim}
## 
##  Pearson's product-moment correlation
## 
## data:  df$p_vote_precise and df$p_osvote_precise
## t = 1.7212, df = 3, p-value = 0.1837
## alternative hypothesis: true correlation is not equal to 0
## 95 percent confidence interval:
##  -0.4691283  0.9785779
## sample estimates:
##      cor 
## 0.704889
\end{verbatim}

\begin{Shaded}
\begin{Highlighting}[]
\CommentTok{\#spearman}
\FunctionTok{cor.test}\NormalTok{(df}\SpecialCharTok{$}\NormalTok{p\_vote\_precise, df}\SpecialCharTok{$}\NormalTok{p\_osvote\_precise, }\AttributeTok{method =} \StringTok{"spearman"}\NormalTok{)}
\end{Highlighting}
\end{Shaded}

\begin{verbatim}
## 
##  Spearman's rank correlation rho
## 
## data:  df$p_vote_precise and df$p_osvote_precise
## S = 10, p-value = 0.45
## alternative hypothesis: true rho is not equal to 0
## sample estimates:
## rho 
## 0.5
\end{verbatim}

\end{document}
